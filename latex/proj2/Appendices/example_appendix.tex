\section{Appendix}\label{sec:appendix}

\subsection{\mintinline{sh}{src/prime.rs}}
\begin{minted}{rust}
use rand::{thread_rng, Rng};
use rug::integer::IsPrime;
use rug::rand::RandState;
use rug::{Complete, Integer};
use std::ops::Range;
use std::fs;
use std::io::{BufReader, BufRead, Write};
use std::str::FromStr;
use crate::prime_statistics;
use crate::random::{randint_bits_odd, randint_bits};
use crate::integer_computations::pow_rug;


pub fn generate_primes() {
    println!("Opening file ./primes");
    // Rewrites file.
    let mut file = fs::OpenOptions::new()
        .write(true)
        .truncate(true)
        .open("primes")
        .unwrap();
    
    
    let mut p: Integer = Integer::ONE.clone();
    let number_of_primes = 1000;
    let space_of_primes = 100000;

    println!("Starting prime generating...");
    for _ in 0..number_of_primes {
        for _ in 0..space_of_primes {
            p.next_prime_mut();
        }
        if let Err(e) = writeln!(file, "{}", &p) {
            eprintln!("Couldn't write to file: {}", e)
        }
    }
}


pub fn generate_non_primes() {
    println!("Opening file ./non-primes");
    // Rewrites file.
    let mut file = fs::OpenOptions::new()
        .create(true)
        .write(true)
        .truncate(true)
        .open("non-primes")
        .unwrap();

    let primes = fs::File::open("primes").expect("File should exist");
    let reader = BufReader::new(primes);

    println!("Generating list of non-primes...");
    for line in reader.lines() {
        let mut rng = thread_rng();
        let p: Integer = Integer::from_str(&line.unwrap()).expect("All entries of file should be numbers"); 
        let mut n: Integer = (&p + rng.gen_range::<usize, Range<usize>>(1..1000)).complete();
        while n.is_probably_prime(30) != IsPrime::No {
            n = (&p + rng.gen_range::<usize, Range<usize>>(1..1000)).complete();
        }

        if let Err(e) = writeln!(file, "{}", &n) {
            eprintln!("Couldn't write to file: {}", e);
        }
    }
}


pub fn generate_small_primes(n: usize) {
    println!("Opening file ./small-primes");
    let mut file = fs::OpenOptions::new()
        .create(true)
        .write(true)
        .truncate(true)
        .open("small-primes")
        .unwrap();

    let mut p: Integer = Integer::from(2);
    
    println!("Generating list of small primes...");
    while p < n {
        if let Err(e) = writeln!(file, "{}", &p) {
            eprintln!("Couldn't write to file: {}", e);
        }
        p.next_prime_mut();
    }
}


pub fn is_likely_prime_with_trial_division(candidate: &Integer, n: usize, bound: usize) -> bool {
    if bound == 0 {
        return rabin_miller_is_prime(candidate, n);
    }
    let small_primes = fs::File::open("small-primes").expect("small-primes file should have been generated");
    let reader = BufReader::new(small_primes);
    for prime in reader.lines() {
        let p: Integer = Integer::from_str(&prime.unwrap()).expect("All entries of small-primes should be integers.");
        if p > bound {
            break;
        }
        if candidate > &p && candidate%p == 0 {
            return false
        }         
    }

    rabin_miller_is_prime(candidate, n)
}


pub fn fermat_is_prime(n: &Integer, reps: usize) -> bool {
    let mut rng = RandState::new();
    for _ in 0..reps {
        let a = Integer::ONE + (n-Integer::ONE.clone()).random_below_ref(&mut rng).complete();
        if pow_rug(&a, &(n-Integer::ONE).complete(), &n) != 1 {
            return false
        }
    }
    true
}


// For n an odd prime with n-1 = 2^s * r with r odd and a in [1, n-1] we have: 
//      a^r = 1 (mod n)    or    a^(2^j * r) = -1 (mod n), for j in [0, s-1]
pub fn rabin_miller_is_prime(n: &Integer, reps: usize) -> bool {
    if *n == 2 {
        return true;
    } else if *n == 3 {
        return true;
    }

    let mut rng = RandState::new();
    let mut r: Integer = n.clone() - Integer::ONE;
    let mut s = 0;
    while !r.get_bit(0) {
        r = r >> 1;
        s += 1;
    }

    for _ in 0..reps {
        let a = Integer::from(2) + (n-Integer::from(4)).random_below(&mut rng);
        let mut y = pow_rug(&a, &r, n);

        if &y != Integer::ONE && y != (n-Integer::ONE).complete() {
            let mut j = 1;

            while j <= s-1 && y != (n-Integer::ONE).complete() {
                y = pow_rug(&y, &Integer::from(2), n);
                if y == 1 {
                    return false;
                }
                j += 1;
            }

            if y != (n-Integer::ONE).complete() {
                return false;
            }
        }
    }

    true
}


pub fn find_prime_with_bit_length(bits: usize, t: usize) -> Integer {
    let mut p: Integer = randint_bits_odd(bits);
    while !rabin_miller_is_prime(&p, t) {
        p = randint_bits_odd(bits);
    }
    p
}


pub fn find_prime_with_bit_length_using_trial_division(bits: usize, t: usize, bound: usize) -> Integer {
    let mut p: Integer = randint_bits_odd(bits);
    while !is_likely_prime_with_trial_division(&p, t, bound) {
        p = randint_bits_odd(bits);
    }
    p
}


pub fn find_prime_with_bit_length_using_interval(bits: usize, d: usize, t: usize, bound: usize) -> Option<Integer> {
    let mut n = randint_bits(bits);
    if is_likely_prime_with_trial_division(&n, t, bound) {
        return Some(n)
    }
    for _ in 0..d {
        n += 1;
        if is_likely_prime_with_trial_division(&n, t, bound) {
            return Some(n)
        }
    }

    None
}


pub fn find_prime_in_interval_with_sieving(a: &Integer, d: usize, t: usize, bound: usize) -> Option<Integer> {
    let small_primes = fs::File::open("small-primes").expect("small-primes file should have been generated");
    let reader = BufReader::new(small_primes);

    let mut vec: Vec<bool> = vec![true; d];
    let mut capacity = d;

    for prime in reader.lines() {
        let p = prime.expect("Line should exist").parse::<usize>().expect("Lines should be primes");
        if p > bound {
            break;
        }

        let off_set: usize = (p - (a%p).complete()).to_usize().expect("Should be small enough");
        let mut count = 0;
        loop {
            let index: usize = count*p + &off_set;
            if index >= d {
                break;
            }

            if vec[index] {
                vec[index] = false;
                capacity -= 1;
            }
            count += 1;
        }
    }

    let mut sieving_vec: Vec<usize> = Vec::with_capacity(capacity);   

    for (i, j) in vec.iter().enumerate() {
        if *j {
            sieving_vec.push(i);
        }
    }

    if sieving_vec.len() == 0 {
        return None
    }

    let mut rng = thread_rng();
    let mut index = rng.gen_range(0..capacity);
    let mut p: Integer = (a + sieving_vec[index]).into();

    while !rabin_miller_is_prime(&p, t) {
        sieving_vec.remove(index);
        if sieving_vec.len() == 0 {
            return None
        }
        capacity -= 1;
        p = (a + sieving_vec[rng.gen_range(0..capacity)]).into();
        index = rng.gen_range(0..capacity);
    }
    Some(p)
}


pub fn find_prime_with_bit_length_using_sieving(bits: usize, t: usize, bound: usize) -> Integer {
    if bound == 0 {
        find_prime_with_bit_length(bits, t);
    }
    let probability = 0.95;
    let d = prime_statistics::approx_width_in_random_interval_search(bits, probability);
    
    loop {
        let a = randint_bits(bits);
        if let Some(p) = find_prime_in_interval_with_sieving(&a, d, t, bound) {
            return p
        }
    }
}



#[cfg(test)]
mod test {
    use std::{io::{BufRead, BufReader}, str::FromStr};

    use super::*;

    #[test]
    fn test_primality_test_for_primes() {
        let file = fs::File::open("primes").unwrap();
        let reader = BufReader::new(file);
        let t = 30;
        let bound = 2000;

        for line in reader.lines() {
            let p: Integer = Integer::from_str(&line.unwrap()).unwrap();
            assert_eq!(true, is_likely_prime_with_trial_division(&p, t, bound), "Identified {} as non-prime", &p);
            assert_eq!(true, fermat_is_prime(&p, t), "Identified {} as non-prime using Fermat-test", &p);
            assert_eq!(true, rabin_miller_is_prime(&p, t), "Identified {} as non-prime using Rabin-Miller", &p);
        }
    }

    #[test]
    fn test_primality_test_for_non_primes() {
        let file = fs::File::open("non-primes").unwrap();
        let reader = BufReader::new(file);
        let t = 30;
        let bound = 2000;

        for line in reader.lines() {
            let n: Integer = Integer::from_str(&line.unwrap()).unwrap();
            assert_eq!(false, is_likely_prime_with_trial_division(&n, t, bound), "Identified {} as prime", &n);
            assert_eq!(false, fermat_is_prime(&n, t), "Identified {} as prime using Fermat-test", &n);
            assert_eq!(false, rabin_miller_is_prime(&n, t), "Identified {} as prime using Rabin-Miller", &n);
        }
    }

    #[test]
    fn test_find_prime_with_bit_length() {
        let reps = 25;
        let t = 30;
        let bound_td = 200;
        let bound_sieving = 30;
        let mut rng = thread_rng();
        for _ in 0..reps {
            let bits = rng.gen_range(5..200);
            let p = find_prime_with_bit_length(bits, t);
            assert!(p.is_probably_prime(t as u32) != IsPrime::No, "Found {} as prime", &p);
            assert_eq!(p.significant_bits(), bits as u32);
        }

        for _ in 0..reps {
            let bits = rng.gen_range(5..200);
            let trial_division = find_prime_with_bit_length_using_trial_division(bits, t, bound_td);
            assert!(trial_division.is_probably_prime(t as u32) != IsPrime::No, "Found {} as prime with trial_division", &trial_division);
            assert_eq!(trial_division.significant_bits(), bits as u32);
        }

        for _ in 0..reps {
            let bits = rng.gen_range(5..200);
            let sieving = find_prime_with_bit_length_using_sieving(bits, t, bound_sieving);
            assert!(sieving.is_probably_prime(t as u32) != IsPrime::No, "Found {} as prime with sieving", &sieving);
            assert_eq!(sieving.significant_bits(), bits as u32);
        }
    }
}
\end{minted}


\subsection{\mintinline{sh}{src/prime_statistics.rs}}
\begin{minted}{rust}
use crate::prime::{find_prime_in_interval_with_sieving, find_prime_with_bit_length, find_prime_with_bit_length_using_sieving, find_prime_with_bit_length_using_trial_division, generate_small_primes, is_likely_prime_with_trial_division, rabin_miller_is_prime};
use crate::random::{randint_bits, randint_bits_odd};
use rug::Integer;
use std::time::Instant;
use std::io::{BufRead, BufReader, Write};
use std::fs;
use std::str::FromStr;


pub fn count_candidates_in_find_prime_with_bit_length(bits: usize, t: usize, bound: usize) -> usize {
    let mut p: Integer = randint_bits_odd(bits);
    let mut count = 1;
    while !is_likely_prime_with_trial_division(&p, t, bound) {
        count += 1;
        p = randint_bits_odd(bits);
    }
    count
}


pub fn count_candidates_in_find_prime_with_bit_length_avg(bits: usize, n: usize, t: usize, bound: usize) -> f64 {
    let mut count = 0;
    for _ in 0..n {
        count += count_candidates_in_find_prime_with_bit_length(bits, t, bound);
    }
    (count as f64)/(n as f64)
}


pub fn width_in_random_interval_search(bits: usize, probability: f64) -> usize {
    let mut d = 1;
    while (1f64 - 1f64/((bits as f64)*2f64.ln())).powf(d as f64) > 1f64 - probability {
        d += 1;
    }
    d
}


pub fn approx_width_in_random_interval_search(bits: usize, probability: f64) -> usize {
    ( - probability/( (1f64 - 1f64/((bits as f64)*2f64.ln())) ).ln() ) as usize
}


pub fn expected_candidates(bits: usize) -> f64 {
    (bits as f64)*2f64.ln()
}


pub fn expected_candidates_with_filtration(bits: usize, bound: usize) -> f64 {
    let small_primes = fs::File::open("small-primes").expect("small-primes file should have been generated");
    let reader = BufReader::new(small_primes);
    let mut prod: f64 = 1.;
    
    for prime in reader.lines() {
        let p: Integer = Integer::from_str(&prime.unwrap()).expect("All entries of small-primes should be integers.");
        if p < bound {
            prod *= (p.to_f64()-1.)/p.to_f64();
        }
    }

    (bits as f64)*2f64.ln()*prod
}


pub fn time_finding_primes() {
    let bits = 300;
    let loops = 80;
    let t = 20;

    let mut number_of_primes_for_trial_division = 0;
    let increase = 50;
    let number_of_increases = 40;
    let mut timing_vector = Vec::with_capacity(number_of_increases);


    for _ in 0..number_of_increases {
        number_of_primes_for_trial_division += increase;
        generate_small_primes(number_of_primes_for_trial_division);
        let now = Instant::now();
        for _ in 0..loops {
            find_prime_with_bit_length(bits, t);
        }
        let elapsed = now.elapsed()/loops;
        timing_vector.push((number_of_primes_for_trial_division, elapsed.as_micros()))
    }

    for (num_of_precomputations, elapsed_as_millis) in timing_vector.iter() {
        println!("{}, {}", num_of_precomputations, elapsed_as_millis);
    }
}


pub fn generate_list_of_composite_until_prime(bits: usize, append: bool) {
    println!("Opening file ./trial-division");
    let mut file = fs::OpenOptions::new()
        .write(true)
        .truncate(!append)
        .append(append)
        .open("trial-division")
        .unwrap();

    let t = 30;

    loop {
        let p = randint_bits(bits);

        if let Err(e) = writeln!(file, "{}", &p) {
            eprintln!("Couldn't write to file: {}", e)
        }
        
        if rabin_miller_is_prime(&p, t) {
            break;
        }
    }
}


pub fn generate_random_numbers(bits: usize, number: usize) {
    let mut file = fs::OpenOptions::new()
        .write(true)
        .truncate(true)
        .open("random-numbers")
        .unwrap();

    for _ in 0..number {
        let a = randint_bits(bits);
        if let Err(e) = writeln!(file, "{}", &a) {
            eprintln!("Couldn't write to file: {}", e)
        }
    }
}


pub fn find_good_number_of_precomputed_primes(bits: usize) {
    let increment = 50;
    let stop = 1000;
    let loops = 40;
    let t = 30;
    let security = 5;

    generate_list_of_composite_until_prime(bits, false);
    for _ in 0..security {
        generate_list_of_composite_until_prime(bits, true);
    }

    let mut bound = 0;

    while bound <= stop {
        let now = Instant::now();
        for _ in 0..loops {
            let list_until_prime = fs::File::open("trial-division").expect("trial-division file should have been generated");
            let reader = BufReader::new(list_until_prime);
            for prime in reader.lines() {
                let p = Integer::from_str(&prime.expect("Line exists")).expect("Every line should be an integer");
                is_likely_prime_with_trial_division(&p, t, bound);
            }
        }
        let elapsed = now.elapsed()/(loops*security) as u32;
        println!("Bound: {}, Elapsed: {}", bound, elapsed.as_micros());
        bound += increment;
    }
}


pub fn find_good_number_of_primes_for_sieving(bits: usize) {
    let increment = 1000;
    let start = 2000;
    let stop = 17000;
    let loops = 40;
    let t = 30;
    let number_of_generated = 40;

    let mut bound = start;
    let d = width_in_random_interval_search(bits, 0.95);

    println!("Found width to be {}", d);

    generate_random_numbers(bits, number_of_generated);

    println!("Done generating lists");

    while bound <= stop {
        // println!("Checking loops now");
        let now = Instant::now();
        for _ in 0..loops {
            // find_prime_with_bit_length_using_sieving(bits, t, bound);
            let list_until_prime = fs::File::open("random-numbers").expect("trial-division file should have been generated");
            let reader = BufReader::new(list_until_prime);
            for number in reader.lines() {
                let a = Integer::from_str(&number.expect("Line should exist")).expect("Lines should be integers");
                // println!("Checking for a={}", a);
                find_prime_in_interval_with_sieving(&a, d, t, bound);
            }
        }
        let elapsed = now.elapsed()/(loops*number_of_generated) as u32;
        println!("Bound: {}, Elapsed: {}", bound, elapsed.as_micros());
        bound += increment;
    }
}


pub fn time_finding_primes_trial_division_vs_sieving() {
    let bits = 500;
    let loops = 1000;
    let bound_td = 150;
    let bound_sieving = 12000;
    let t = 30;

    let now = Instant::now();
    for _ in 0..loops {
        find_prime_with_bit_length_using_trial_division(bits, t, bound_td);
    }
    let elapsed_td = now.elapsed().as_micros()/loops;
    println!("trial-division: {}", elapsed_td);

    let now = Instant::now();
    for _ in 0..loops {
        find_prime_with_bit_length_using_sieving(bits, t, bound_sieving);
    }
    let elapsed_sieving = now.elapsed().as_micros()/loops;

    println!("sieving: {}", elapsed_sieving);
}
\end{minted}


\subsection{\mintinline{sh}{src/main.rs}}
\begin{minted}{rust}
// This is in addition to computational algebra a personal learning experience with rust.
use std::env;

use prime::{find_prime_with_bit_length, find_prime_with_bit_length_using_sieving, find_prime_with_bit_length_using_trial_division};

pub mod algebraic_structure;
pub mod tests;
pub mod integer_computations;
pub mod prime;
pub mod prime_statistics;
pub mod random;


fn options_handler(args: Vec<String>) -> bool {
    let mut used = false;
    for arg in &args[1..] {
        if arg == "generate-test-lists" {
            prime::generate_primes();
            prime::generate_non_primes();
            used = true;
        } else if arg == "generate-small-primes" {
            let n = 100000;
            prime::generate_small_primes(n);
            used = true;
        } else if arg == "proj1" {
            let loops = 10;
            let naive_square_points: usize = 20;
            let naive_points: usize = 50;
            let square_points: usize = 50;
            tests::plot_timing_naive_square(naive_square_points, loops).expect("Should not fail");
            tests::plot_timing_naive(naive_points, loops).expect("Should not fail");
            tests::plot_timing_square(square_points, loops).expect("Should not fail");
            used = true;
        } else if arg == "proj2" {
            continue;
        } else {
            println!("Argument {} is not valid.", arg);
        }
    }
    used
}


fn main() {
    if !options_handler(env::args().collect()) {
        main_without_arg();
    }
}


fn main_without_arg() {
    // prime_statistics::find_good_number_of_precomputed_primes(500);
    // prime_statistics::find_good_number_of_primes_for_sieving(500);
    prime_statistics::time_finding_primes_trial_division_vs_sieving();
}
\end{minted}


\subsection{\mintinline{sh}{src/random.rs}}
\begin{minted}{rust}
use rand::{thread_rng, Rng};
use rug::Integer;


pub fn randint_bits(bits: usize) -> Integer {
    let mut n = Integer::from(1);
    let mut rng = thread_rng();
    for _ in 1..bits {
        n <<= 1;
        n += rng.gen_range(0..=1);
    }
    n
}


pub fn randint_bits_odd(bits: usize) -> Integer {
    let mut n = randint_bits(bits-1);
    n <<= 1;
    n += 1;
    n
}


pub fn randint_digits(digits: usize) -> Integer {
    let mut rng = thread_rng();
    let mut n: Integer = Integer::from(rng.gen_range(1..=9));

    for _ in 0..digits {
        n *= 10;
        n += rng.gen_range(0..=9);
    }
    n
}
\end{minted}


\subsection{\mintinline{sh}{src/tests.rs}}
\begin{minted}{rust}
use std::{sync::Arc, time::Duration};
use std::time::Instant;
use crate::{algebraic_structure::{finite_field::FiniteField, Element}, integer_computations::naive_pow};
use rug::integer::IsPrime;
use rug::ops::PowAssign;
use rug::Integer;
use plotters::prelude::*;
use plotters::coord::combinators::IntoLogRange;
use crate::random::{randint_bits, randint_digits};


pub fn check_timing_against_rug(a: &Integer, b: &Integer, p: &Integer) -> (Duration, Duration) {
    let f = Arc::new(FiniteField::new(p.clone()).unwrap());
    let a_elem = Element::new(f.clone(), a.clone());
    let b_exp = b.clone();

    let now = Instant::now();
    a_elem.pow(&b_exp);
    let elapsed_elem = now.elapsed();

    let now = Instant::now();
    a.clone().pow_mod(&b, &p).unwrap();
    let elapsed = now.elapsed();

    (elapsed_elem, elapsed)
}


fn check_timing_naive(a: &Integer, b: &Integer, p: &Integer, n: usize) -> Duration {
    let now = Instant::now();
    for _ in 0..n {
        naive_pow(a, b, p);
    }
    now.elapsed()/50
}


fn check_timing_square(a: &Integer, b: &Integer, p: &Integer, n: usize) -> Duration {
    let f = Arc::new(FiniteField::new(p.clone()).unwrap());
    let a_elem = Element::new(f.clone(), a.clone());
    
    let now = Instant::now();
    for _ in 0..n {
        a_elem.pow(b);
    }
    now.elapsed()/50
}


pub fn check_timing_against_naive(a: &Integer, b: &Integer, p: &Integer, n: usize) -> (Duration, Duration) {
    (check_timing_naive(a, b, p, n), check_timing_square(a, b, p, n))
}


pub fn plot_timing_naive_square(n: usize, m: usize) -> Result<(), Box<dyn std::error::Error>> {
    let mut q = Integer::from(17);
    let mut p = Integer::from(2);
    p.pow_assign(127);
    p = p-1;
    assert!(p.is_probably_prime(20) != IsPrime::No);
    // let mut rng = RandState::new();
    let mut naive_vec: Vec<(u128, u128)> = Vec::new();
    let mut square_vec: Vec<(u128, u128)> = Vec::new();
    let mut max_time_naive = 0;
    // let mut last_bit_number = p.significant_bits();
    for _ in 0..n {
        // while p.significant_bits() == last_bit_number {
        //     p.next_prime_mut();
        // }
        // last_bit_number = p.significant_bits();
        for _ in 0..5 {
            q.next_prime_mut();
        }

        let a = randint_digits(q.to_string().len());
        // let a = p.clone() - Integer::ONE.clone();
        // let b = randint_digits(p.significant_digits::<usize>());
        let b = q.clone();
        let (elapsed_naive, elapsed_square) = check_timing_against_naive(&a, &b, &p, m);
        naive_vec.push((q.to_u128().unwrap(), elapsed_naive.as_nanos()));
        square_vec.push((q.to_u128().unwrap(), elapsed_square.as_nanos()));
        if elapsed_naive.as_nanos() > max_time_naive {
            max_time_naive = elapsed_naive.as_nanos();
        }
    }

    // for i in &square_vec {
    //     println!("{}", i.1);
    // }

    let root = SVGBackend::new("../latex/proj1/images/naive-square.svg", (600, 400)).into_drawing_area();
    root.fill(&WHITE)?;

    let mut chart = ChartBuilder::on(&root)
        .caption("Runtime of Naive vs Square-Multiply", ("computer-modern", 30).into_font())
        .margin(40)
        .x_label_area_size(30)
        .y_label_area_size(50)
        .build_cartesian_2d(0..q.to_u128().unwrap_or(u128::MAX), 0..max_time_naive)?;

    chart.configure_mesh()
        .x_desc("Prime number")
        .x_label_style(("computer-modern", 12).into_font())
        .y_desc("Nanoseconds")
        .y_label_style(("computer-modern", 12).into_font())
        .draw()?;

    chart
        .draw_series(PointSeries::of_element(
            // (-50..=50).map(|x| x as f32 / 50.0).map(|x| (x, x * x)),
            naive_vec,
            3,
            &RED,
            &|c, s, st| {
                return EmptyElement::at(c)    // We want to construct a composed element on-the-fly
                + Circle::new((0,0),s,st.filled()) // At this point, the new pixel coordinate is established
                // + Text::new(format!("{:?}", c), (10, 0), ("sans-serif", 10).into_font());
            },
        ))?
        .label("Naive")
        .legend(|(x, y)| Circle::new((x, y), 3, RED.filled()));
        // .legend(|(x, y)| PathElement::new(vec![(x, y), (x + 20, y)], &RED));

    chart
        .draw_series(PointSeries::of_element(
            // (-50..=50).map(|x| x as f32 / 50.0).map(|x| (x, x * x)),
            square_vec,
            3,
            &BLUE,
            &|c, s, st| {
                return EmptyElement::at(c)    // We want to construct a composed element on-the-fly
                + Circle::new((0,0),s,st.filled()) // At this point, the new pixel coordinate is established
                // + Text::new(format!("{:?}", c), (10, 0), ("sans-serif", 10).into_font());
            },
        ))?
        .label("Square-Multiply")
        // .legend(|(x, y)| PathElement::new(vec![(x, y), (x + 20, y)], &BLUE));
        .legend(|(x, y)| Circle::new((x, y), 3, BLUE.filled()));

    chart
        .configure_series_labels()
        .label_font(("computer-modern", 12).into_font())
        .background_style(&WHITE.mix(0.8))
        .border_style(&BLACK)
        .legend_area_size(12)
        .draw()?;

    root.present()?;


    Ok(())
}


pub fn plot_timing_naive(n: usize, m: usize) -> Result<(), Box<dyn std::error::Error>> {
    let mut p = Integer::ONE.clone();
    let mut q = Integer::from(2);
    q.pow_assign(127);
    q = q-1;
    assert!(q.is_probably_prime(20) != IsPrime::No);
    // let mut rng = RandState::new();
    let mut naive_vec: Vec<(u128, u128)> = Vec::new();
    let mut max_time_naive = 0;
    // let mut last_bit_number = p.significant_bits();
    for _ in 1..=n {
        // while p.significant_bits() == last_bit_number {
        //     p.next_prime_mut();
        // }
        // last_bit_number = p.significant_bits();
        for _ in 0..400 {
            p.next_prime_mut();
        }
        // p = randint_digits(i as usize).next_prime();

        let a = randint_digits(p.to_string().len());
        // let a = p.clone() - Integer::ONE.clone();
        // let b = randint_digits(p.to_string().len());
        let b = p.clone();
        println!("{}", b);
        let elapsed_naive = check_timing_naive(&a, &b, &q, m);
        naive_vec.push((p.to_u128().unwrap(), elapsed_naive.as_micros()));
        if elapsed_naive.as_micros() > max_time_naive {
            max_time_naive = elapsed_naive.as_micros();
        }
        // println!("{}, {}, {}", &p, elapsed_square.as_nanos(), elapsed_naive.as_nanos());
    }

    let root = SVGBackend::new("../latex/proj1/images/naive.svg", (600, 400)).into_drawing_area();
    root.fill(&WHITE)?;

    let mut chart = ChartBuilder::on(&root)
        .caption("Runtime of Naive Exponentiation", ("computer-modern", 30).into_font())
        .margin(40)
        .x_label_area_size(30)
        .y_label_area_size(50)
        .build_cartesian_2d(0..p.to_u128().unwrap_or(u128::MAX), 0..max_time_naive)?;

    chart.configure_mesh()
        .x_desc("Prime number")
        .x_label_style(("computer-modern", 12).into_font())
        .y_desc("Microseconds")
        .y_label_style(("computer-modern", 12).into_font())
        .draw()?;

    chart
        .draw_series(PointSeries::of_element(
            // (-50..=50).map(|x| x as f32 / 50.0).map(|x| (x, x * x)),
            naive_vec,
            3,
            &RED,
            &|c, s, st| {
                return EmptyElement::at(c)    // We want to construct a composed element on-the-fly
                + Circle::new((0,0),s,st.filled()) // At this point, the new pixel coordinate is established
                // + Text::new(format!("{:?}", c), (10, 0), ("sans-serif", 10).into_font());
            },
        ))?;
        // .label("Naive approach")
        // .legend(|(x, y)| PathElement::new(vec![(x, y), (x + 20, y)], &RED));
        // .legend(|(x, y)| PathElement::new(vec![(x, y), (x + 20, y)], &RED));

    // chart
    //     .configure_series_labels()
    //     .background_style(&WHITE.mix(0.8))
    //     .border_style(&BLACK)
    //     .draw()?;

    root.present()?;
    
    Ok(())
}


pub fn plot_timing_square(n: usize, m: usize) -> Result<(), Box<dyn std::error::Error>> {

    let mut q = Integer::from(2);
    q.pow_assign(127);
    q = q-1;
    assert!(q.is_probably_prime(20) != IsPrime::No);
    let mut p = Integer::ONE.clone();
    // let mut rng = RandState::new();
    let mut square_vec: Vec<(u64, u64)> = Vec::new();
    let mut max_time_square = 0;
    // let mut last_bit_number = p.significant_bits();
    for i in 1..=n {
        // while p.significant_bits() == last_bit_number {
        //     p.next_prime_mut();
        // }
        // last_bit_number = p.significant_bits();
        // for _ in 0..400 {
        //     p.next_prime_mut();
        // }
        p = randint_bits(i as usize).next_prime();

        let a = randint_digits(p.to_string().len());
        // let b = randint_digits(p.significant_digits::<usize>());
        // let a = p.clone() - Integer::ONE.clone();
        let b = p.clone();
        let elapsed_square = check_timing_square(&a, &b, &q, m);
        square_vec.push((p.to_u64().unwrap(), elapsed_square.as_nanos() as u64));
        if elapsed_square.as_nanos() > max_time_square {
            max_time_square = elapsed_square.as_nanos();
        }
        // println!("{}, {}, {}", &p, elapsed_square.as_nanos(), elapsed_naive.as_nanos());
    }


    let root = SVGBackend::new("../latex/proj1/images/square.svg", (600, 400)).into_drawing_area();
    root.fill(&WHITE)?;

    let mut chart = ChartBuilder::on(&root)
        .caption("Runtime of Square-Multiply", ("computer-modern", 30).into_font())
        .margin(30)
        .x_label_area_size(30)
        .y_label_area_size(50)
        .build_cartesian_2d((0..p.to_u64().unwrap_or(u64::MAX)).log_scale(), 0..max_time_square as u64)?;

    chart.configure_mesh()
        .x_desc("Prime number")
        .x_label_formatter(&|x| format!("10^{}", x.ilog10()))
        .x_label_style(("computer-modern", 12).into_font())
        .y_desc("Nanoseconds")
        .y_label_style(("computer-modern", 12).into_font())
        .draw()?;

    chart
        .draw_series(PointSeries::of_element(
            // (-50..=50).map(|x| x as f32 / 50.0).map(|x| (x, x * x)),
            square_vec,
            3,
            &BLUE,
            &|c, s, st| {
                return EmptyElement::at(c)    // We want to construct a composed element on-the-fly
                + Circle::new((0,0),s,st.filled()) // At this point, the new pixel coordinate is established
                // + Text::new(format!("{:?}", c), (10, 0), ("sans-serif", 10).into_font());
            },
        ))?;
        // .label("Square approach")
        // .legend(|(x, y)| PathElement::new(vec![(x, y), (x + 20, y)], &BLUE));

    // chart
    //     .configure_series_labels()
    //     .background_style(&WHITE.mix(0.8))
    //     .border_style(&BLACK)
    //     .draw()?;

    root.present()?;

    Ok(())
}


#[cfg(test)]
mod tests {
    use super::*;
    use crate::algebraic_structure::Element;
    use crate::algebraic_structure::finite_field::MultiplicativeGroup;
    use rug::{Complete, rand::RandState};
    use std::sync::Arc;

    #[test]
    fn test_algebraic_structure_arithmetic() {
        let n: u32 = 200;
        let mut prime: Integer = Integer::from(2);
        let mut rng = RandState::new();

        for _ in 2..n {
            let p = prime.clone();
            let f: Arc<FiniteField> = Arc::new(FiniteField::new(p.clone()).unwrap());
            let a_rand: Integer = Integer::from(rng.bits(32));
            let b_rand: Integer = Integer::from(rng.bits(32));
            let a = Element::new(f.clone(), a_rand.clone());
            let b = Element::new(f.clone(), b_rand.clone());

            assert_eq!(a.get_rep(), &(&a_rand % &p).complete());
            assert_eq!(b.get_rep(), &(&b_rand % &p).complete());

            let added = (a.add_ref(&b)).get_rep().clone();
            let subtracted = (a.sub_ref(&b)).get_rep().clone();
            let multiplied = (a.mul_ref(&b)).get_rep().clone();

            let added_check: Integer = ((&a_rand % &p).complete() + (&b_rand % &p).complete()) % &p;
            let sub_check: Integer = ((&a_rand % &p).complete() - (&b_rand % &p).complete()).modulo(&p);
            let mul_check: Integer = ((a_rand % &p) * (b_rand % &p)) % &p;

            assert_eq!(added, added_check, "Failed for: {} + {}, and got {} instead of {}", a.get_rep(), b.get_rep(), added, added_check);
            assert_eq!(subtracted, sub_check, "Failed for: {} + {}, and got {} instead of {}", a.get_rep(), b.get_rep(), subtracted, sub_check);
            assert_eq!(multiplied, mul_check, "Failed for: {} * {}, and got {} instead of {}", a.get_rep(), b.get_rep(), multiplied, mul_check);

            let g: Arc<MultiplicativeGroup> = Arc::new(MultiplicativeGroup::from_finite_field(&f));

            // To not create zero
            let modification: Integer = Integer::from(rng.bits(32));
            let a_rand: Integer = (Integer::from(rng.bits(32)).modulo(&(&p-&Integer::ONE.clone()).complete())) + 1 + (&p * &modification);
            let b_rand: Integer = (Integer::from(rng.bits(32)).modulo(&(&p-&Integer::ONE.clone()).complete())) + 1 + (&p * modification);

            let a = Element::new(g.clone(), a_rand.clone());
            let b = Element::new(g.clone(), b_rand.clone());

            assert_eq!(a.get_rep(), &(&a_rand % &p).complete(), "Failed representation of {} in Z_{}*,  got {} instead of {}", &a_rand, &p, a.get_rep(), (&a_rand % &p).complete());
            assert_eq!(b.get_rep(), &(&b_rand % &p).complete(), "Failed representation of {} in Z_{}*,  got {} instead of {}", &b_rand, &p, b.get_rep(), (&b_rand % &p).complete());

            let multiplied = a.mul_ref(&b);
            let mul_check: Integer = ((&a_rand % &p).complete() * (&b_rand % &p).complete()) % &p;

            assert_eq!(multiplied.get_rep(), &mul_check);

            let divided = a.div_ref(&b);

            assert_eq!(divided.mul_ref(&b).get_rep(), &(&a_rand % &p).complete(), "Failed division {}/{} in Z_{}*", &a_rand, &b_rand, &p);


            prime.next_prime_mut();
        }
    }

    #[test]
    #[should_panic(expected = "Zero Division")]
    fn test_zero_division() {
        let mut rng = RandState::new();
        let p: Integer = Integer::from(rng.bits(32)).next_prime();
        let a_rand: Integer = Integer::from(rng.bits(32));
        let f: Arc<FiniteField> = Arc::new(FiniteField::new(p).unwrap());
        let a_elem: Element<FiniteField> = Element::new(f.clone(), a_rand);
        let b_elem: Element<FiniteField> = Element::new(f, Integer::ZERO);

        a_elem.div_ref(&b_elem);
    }

    #[test]
    fn test_extended_euclidean() {
        let n: u32 = 200;
        let mut prime: Integer = Integer::from(2);
        let mut rng = RandState::new();

        for _ in 2..n {
            let p = prime.clone();
            let f = Arc::new(FiniteField::new(p.clone()).unwrap());
            let a_rand = (Integer::from(rng.bits(32)).modulo(&(&p-&Integer::ONE.clone()).complete())) + 1 ;
            let a = Element::new(f.clone(), a_rand);

            let a_inv = a.mul_inv();
            assert_eq!(Integer::ONE, a_inv.mul_ref(&a).get_rep(), "We have a: {}, a_inv: {}, p: {}", a.get_rep(), a_inv.get_rep(), p);

            let g = Arc::new(MultiplicativeGroup::from_finite_field(&f));
            let a_rand = (Integer::from(rng.bits(32)).modulo(&(&p-&Integer::ONE.clone()).complete())) + 1 ;
            let a = Element::new(g.clone(), a_rand);
            let a_inv = a.mul_inv();
            assert_eq!(Integer::ONE, a_inv.mul_ref(&a).get_rep(), "We have a: {}, a_inv: {}, p: {}", a.get_rep(), a_inv.get_rep(), p);


            prime.next_prime_mut();

        }
    }


    #[test]
    fn test_exponentiation() {
        let n: u32 = 200;
        let mut rng = RandState::new();
        let mut prime: Integer = Integer::from(2);


        for _ in 2..n {
            let p = prime.clone();
            let f = Arc::new(FiniteField::new(p.clone()).unwrap());

            let a_rand = Integer::from(rng.bits(32));
            let a = Element::new(f.clone(), a_rand.clone());

            let x_rand = Integer::from(rng.bits(32));

            // let a_exp_check = a_rand.exp_residue(&x_rand, &p);
            let a_exp_check = a_rand.pow_mod_ref(&x_rand, &p).unwrap().complete();
            let a_exp = a.pow(&x_rand);

            assert_eq!(a_exp.get_rep(), &a_exp_check);

            prime.next_prime_mut();
        }
    }
}
\end{minted}


\subsection{\mintinline{sh}{src/integer_computations.rs}}
\begin{minted}{rust}
use rug::{Complete, Integer};
use crate::algebraic_structure::{Element, HasRepresentation};


pub fn extended_euclidean_ordered(a: &Integer, b: &Integer) -> (Integer, Integer, Integer) {
    // There have been no attempt to optimize this function using cloning and references
    // efficiently.
    let mut a1: Integer = a.clone();
    let mut b1: Integer = b.clone();

    if b.is_zero() {
        return (a1, Integer::ONE.clone(), Integer::ZERO.clone())
    };

    let mut x2: Integer = Integer::ONE.clone();
    let mut x1: Integer = Integer::ZERO.clone();
    let mut y2: Integer = Integer::ZERO.clone();
    let mut y1: Integer = Integer::ONE.clone();


    while b1 > Integer::ZERO {
        // q = a1/b1;
        let (q, _) = (a1.div_rem_floor_ref(&b1)).complete();
        let r = a1 - &q*&b1;
        let x = x2 - &q*&x1;
        let y = y2 - &q*&y1;
        a1 = b1;
        b1 = r;
        x2 = x1;
        x1 = x;
        y2 = y1;
        y1 = y;
    };

    (a1, x2, y2)
}


pub fn extended_euclidean_to_integers<T: HasRepresentation + Clone>(a: &Element<T>, b: &Element<T>) -> (Integer, Integer, Integer) {
    let a_rep: &Integer = a.get_rep();
    let b_rep: &Integer = b.get_rep();
    if a_rep > b_rep {
        extended_euclidean_ordered(a_rep, b_rep)
    } else {
        extended_euclidean_ordered(b_rep, a_rep)
    }
}


pub fn pow_rug(a: &Integer, b: &Integer, n: &Integer) -> Integer {
    let mut product: Integer = Integer::ONE.clone();
    let mut base = a.clone() % n;
    let mut exponent = b.clone();

    while exponent != 0 {
        if exponent.get_bit(0) {
            product = (product * &base).modulo(n);
        }
        base = base.square().modulo(n);
        exponent = exponent >> 1;
    }
    product
}


pub fn naive_pow(a: &Integer, b: &Integer, n: &Integer) -> Integer {
    let mut product: Integer = Integer::ONE.clone();
    let s = b.to_u64().expect("The number is WAY too high to naively calculate.");
    for _ in 0..s {
        product *= a;
        product %= n;
    }
    product
}
\end{minted}


\subsection{\mintinline{sh}{src/algebraic_structure.rs}}
\begin{minted}{rust}
use rug::Integer;
use std::sync::Arc;
pub mod finite_field;


pub trait HasRepresentation {
    fn make_representation(&self, repr: Integer) -> Integer;
}


pub trait HasMul: HasRepresentation + Clone {
    fn mul(&self, a: &Element<Self>, b: &Element<Self>) -> Element<Self>;
    fn pow(&self, a: &Element<Self>, b: &Integer) -> Element<Self>;
}


pub trait HasAdd: HasRepresentation + Clone {
    fn add(&self, a: &Element<Self>, b: &Element<Self>) -> Element<Self>;
}


pub trait HasSub: HasAdd {
    fn add_inv(&self, a: &Element<Self>) -> Element<Self>;
    fn sub_ref(&self, a: &Element<Self>, b: &Element<Self>) -> Element<Self> {
        self.add(a, &self.add_inv(b))
    }
}


pub trait HasDiv: HasMul {
    fn mul_inv(&self, a: &Element<Self>) -> Element<Self>;
    fn div(&self, a: &Element<Self>, b: &Element<Self>) -> Element<Self> {
        self.mul(a, &self.mul_inv(&b))
    }
}


#[derive(Debug, Clone)]
pub struct Element<T: HasRepresentation + Clone> {
    outer_structure: Arc<T>,
    representation: Integer,
}


impl<T: HasRepresentation + Clone> Element<T> {
    pub fn new(outer_structure: Arc<T>, repr: Integer) -> Element<T> {
        let representation: Integer = outer_structure.make_representation(repr);
        Element { outer_structure, representation }
    }


    pub fn get_outer_structure(&self) -> Arc<T> {
        self.outer_structure.clone()
    }


    pub fn get_rep(&self) -> &Integer {
        &self.representation
    }
}


impl<T: HasMul> Element<T> {
    pub fn mul_ref(&self, _rhs: &Element<T>) -> Element<T> {
        self.outer_structure.mul(&self, _rhs)
    }

    pub fn pow(&self, a: &Integer) -> Element<T> {
        self.get_outer_structure().pow(self, a)
    }
}


impl<T: HasAdd> Element<T> {
    pub fn add_ref(&self, _rhs: &Element<T>) -> Element<T> {
        self.outer_structure.add(&self, _rhs)
    }
}


impl<T: HasDiv> Element<T> {
    pub fn mul_inv(&self) -> Element<T> {
        self.outer_structure.mul_inv(&self)
    }

    pub fn div_ref(&self, b: &Element<T>) -> Element<T> {
        self.outer_structure.div(&self, b)
    }
}


impl<T: HasSub> Element<T> {
    pub fn add_inv(&self) -> Element<T> {
        self.outer_structure.add_inv(&self)
    }

    pub fn sub_ref(&self, b: &Element<T>) -> Element<T> {
        self.outer_structure.sub_ref(&self, b)
    }
}
\end{minted}


\subsection{\mintinline{sh}{src/algebraic_structure/finite_field.rs}}
\begin{minted}{rust}
use crate::algebraic_structure::{Element, HasAdd, HasMul, HasRepresentation, HasSub};
use crate::integer_computations::{extended_euclidean_ordered, extended_euclidean_to_integers, pow_rug};
use rug::ops::SubFrom;
use rug::{integer::IsPrime, Complete, Integer};
use std::sync::Arc;


use super::HasDiv;

#[derive(Debug, Clone)]
pub struct FiniteField {
    // This struct will only consider finite fields isomorphic to Z_p for p prime.
    size: Integer,
}


impl HasRepresentation for FiniteField {
    fn make_representation(&self, repr: Integer) -> Integer {
        repr.modulo(self.mod_num())
    }
}


impl HasMul for FiniteField {
    fn mul(&self, a: &Element<FiniteField>, b: &Element<FiniteField>) -> Element<FiniteField> {
        Element::new(
            a.get_outer_structure(),
            (a.get_rep() * b.get_rep()).complete() % self.mod_num()
        )
    }

    fn pow(&self, a: &Element<Self>, b: &Integer) -> Element<Self> {
        Element::new(
            a.get_outer_structure(),
            pow_rug(a.get_rep(), b, self.mod_num())
        )
    }
}


impl HasAdd for FiniteField {
    fn add(&self, a: &Element<FiniteField>, b: &Element<FiniteField>) -> Element<FiniteField> {
        Element::new(
            a.get_outer_structure().clone(),
            (a.get_rep() + b.get_rep()).complete() % self.mod_num()
        )
    }
}


impl HasDiv for FiniteField {
    fn mul_inv(&self, a: &Element<Self>) -> Element<Self> {
        if a.get_rep().is_zero() {
            panic!("Zero Division");
        }
        let (_, _, y) = extended_euclidean_ordered(self.mod_num(), a.get_rep());
        Element::new(
            a.get_outer_structure(),
            y
        )
    }
}


impl HasSub for FiniteField {
    fn add_inv(&self, a: &Element<Self>) -> Element<Self> {
        let mut representation: Integer = a.get_rep().clone();
        representation.sub_from(self.mod_num());
        Element {
            outer_structure: a.get_outer_structure().clone(),
            representation,
        }
    }
}


impl FiniteField {
    pub fn new(size: Integer) -> Option<FiniteField> {
        if size.is_probably_prime(30) != IsPrime::No {
            Some(FiniteField {
                size,
            })
        } else {
            None
        }
    }

    pub fn one(self) -> Element<FiniteField> {
        Element {
            outer_structure: Arc::new(self),
            representation: Integer::ONE.clone(),
        }
    }

    pub fn zero(self) -> Element<FiniteField> {
        Element {
            outer_structure: Arc::new(self),
            representation: Integer::ZERO.clone(),
        }
    }

    pub fn get_size(&self) -> Integer {
        self.size.clone()
    }

    fn mod_num(&self) -> &Integer {
        &self.size
    }


    pub fn extended_euclidean(&self, a: &Element<FiniteField>, b: &Element<FiniteField>) -> (Element<FiniteField>, Element<FiniteField>, Element<FiniteField>) {
        let (d, x, y) = extended_euclidean_to_integers(a, b);
        (
            Element::new(
                a.get_outer_structure(),
                d
            ),
            Element::new(
                a.get_outer_structure(),
                x
            ),
            Element::new(
                a.get_outer_structure(),
                y
            ),
        )
    }
}


#[derive(Debug, Clone)]
pub struct MultiplicativeGroup {
    mod_num: Integer,
}


impl HasRepresentation for MultiplicativeGroup {
    // Makes representation for creating elements.
    // As 0 is not present in the mutliplicative group, it is assumed you meant identity and will
    // get 1 instead.
    fn make_representation(&self, repr: Integer) -> Integer {
        let representation = repr.modulo(self.mod_num());
        if representation == 0 {
            Integer::ONE.clone()
        } else {
            representation
        }
    }
}


impl HasMul for MultiplicativeGroup {
    fn mul(&self, a: &Element<MultiplicativeGroup>, b: &Element<MultiplicativeGroup>) -> Element<MultiplicativeGroup> {
        Element::new(
            a.get_outer_structure(),
            (a.get_rep() * b.get_rep()).complete() % self.mod_num()
        )
    }

    fn pow(&self, a: &Element<Self>, b: &Integer) -> Element<Self> {
        Element::new(
            a.get_outer_structure(),
            pow_rug(a.get_rep(), b, self.mod_num()),
        )
    }
}


impl MultiplicativeGroup {
    pub fn new(mod_num: Integer) -> MultiplicativeGroup {
        MultiplicativeGroup {
            mod_num,
        }
    }


    pub fn from_finite_field(finite_field: &FiniteField) -> MultiplicativeGroup {
        let mod_num = finite_field.mod_num().clone();
        MultiplicativeGroup {
            mod_num,
        }
    }


    pub fn mod_num(&self) -> &Integer {
        &self.mod_num
    }


    pub fn get_size(&self) -> Integer {
        (self.mod_num() - Integer::ONE).complete()
    }
}


impl HasDiv for MultiplicativeGroup {
    fn mul_inv(&self, a: &Element<Self>) -> Element<Self> {
        let (_, _, y) = extended_euclidean_ordered(self.mod_num(), a.get_rep());
        Element::new(
            a.get_outer_structure(),
            y
        )
    }
}
\end{minted}


\subsection{\mintinline{sh}{Cargo.toml}}
\begin{minted}{toml}
[package]
name = "proj1"
version = "0.1.0"
edition = "2021"

[dependencies]
plotters = "0.3.7"
rand = "0.8.5"
rug = "1.26.1"

[profile.release]
panic = 'abort'
\end{minted}

