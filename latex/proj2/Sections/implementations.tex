\section{Implementation of Prime Search}

\subsection{Miller-Rabin Primality Test}

  To check if a number is prime we will use the Miller-Rabin primality test, which builds on the following principle.
  Given $p$ an odd prime and $p-1 = 2^s r$ for $r$ an odd integer, let $a \in \M{N}$ such that $\operatorname{gcd}(a, n) = 1$.
  Then either $a^r \equiv 1 \pmod{n}$ or $a^{2^j r} \equiv -1 \pmod{n}$ for some $0 \leq j \leq s-1$.
  In fact for an odd composite number $q$, at most $1/4$ of the numbers in the interval $\bbrack{1, n-1}$ satisfies this property.
  So we have an upper bound for the error of the Miller-Rabin primality test of $1/4$.
  Now we can do the test $t$ times choosing some random $a \in \bbrack{1, n-1}$.
  Then we have an upper error bound of $\nbrack{1/4}^t$.
  This leads us to Algorithm~\ref{alg:miller-rabin}.
  
  \begin{algorithm}
    \caption{Miller-Rabin Primality Test}
    \label{alg:miller-rabin}
    \textbf{Input:} $n \leq 3, t \leq 1$. \\
    \textbf{Output:} Prime or Composite to the question ``Is $n$ prime?''
    \begin{algorithmic}[1]
      \State $r \gets n-1$
      \State $s \gets 0$
      \While{last bit of the binary representation of $r$ is 0}
        \State right bitwise shift of $r$
        \State $s \gets s+1$
      \EndWhile

      \For{$i$ from $1$ to $t$}
        \State Choose a random integer $2 \leq a \leq n-2$.
        \State $y \gets a^r \pmod{n}$
        
        \If{$y \neq 1$ and $y \neq n-1$ }
          \State $j \gets 1$
          \While{$j \leq s-1$ and $y \neq n-1$}
            \State $y \gets y^2 \pmod{n}$
            \If{$y=1$}
              \Return Composite
            \EndIf
            \State $j \gets j+1$
          \EndWhile
          \If{$y \neq n-1$}
            \Return Composite
          \EndIf
        \EndIf    
      \EndFor \\
      \Return Prime
    \end{algorithmic}
  \end{algorithm}

  Now we consider the complexity of this algorithm.
  The first while-loop will run $s$ times where $s \leq \log n$.
  The next chunk of the algorithm is one large for-loop, which runs a total of $t$ times.
  Here we do one modular exponentiation which we for our purposes will assume is done in $\O \bbrack{ \nbrack{ \log n }^3}$.
  Then, if we are not done, we start a while-loop where the worst case scenario is $s-1$ loops, where we still have $s \leq \log n$.
  We would then have a worst case scenario of $\O \bbrack{ t \nbrack{ \log n }^3 }$.


\subsection{Finding a Prime}
  We now go back to our first question in Section~\ref{sec:random-search}, and we will use the exact same approach.
  We will pick a random number $a \in S_n = \bbrack{ 2^n, 2^n + 1, \dots, 2^{n+1}-1 }$ and check if $a$ is prime using Miller-Rabin.
  As we have earlier discussed, we will expect testing $n \ln 2$ candidates before finding a prime.
  
