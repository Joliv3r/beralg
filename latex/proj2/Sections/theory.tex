\section{Searching for primes}

\subsection{Random search} \label{sec:random-search}
  We are looking at the problem of finding primes.
  Our goal is to search for a random prime in the set $S_n = \cbrack{ 2^n, 2^n+1, \dots 2^{n+1}-1 }$.
  We consider the approach of randomly selecting a number from $S_n$ and then checking if this is a prime.
  Now to estimate how many primes we need to check we use the prime number theorem, which states that the number of primes $\pi (N)$ in the interval $\bbrack{2, N}$ is found with the distribution

  \begin{equation}
    \pi(N) \sim \frac{N}{\ln N}
    \label{eq:pmt}
  \end{equation}

  Let $\pi_n$ denote the number of prime numbers in $S_n$, then

  \begin{equation}
    \pi_n = \pi\nbrack{ 2^{n+1} } - \pi\nbrack{ 2^{n} } \sim \frac{2^{n+1}}{\ln 2^{n+1}} - \frac{2^{n}}{\ln 2^{n}}
    = \frac{ 2^{n} }{n \ln 2}.
    \label{eq:primes-in-sn}
  \end{equation}

  Now imagine that we pick a random integer $q \in S_n$, then the probability of $q$ being a prime is then

  \begin{equation}
    \mathcal{Q}_n = \frac{\pi_n}{2^n} = \frac{1}{n\ln 2}
    \label{eq:probability-of-prime}
  \end{equation}

  So the expected number of choices before we find a prime is $n \ln 2$, which we can derive by looking at this as a geometric probability distribution.

  So as an example, we choose $n = 500$.
  Then we can expect to chose $500 \cdot \ln 2 = 347$ candidates before we find a prime. 


\subsection{Search with interval}

  We can also consider another method for finding a random prime.
  Take a random number $a \in S_n$ and use some fixed $d \in \M{N}$ not too large such that we are reasonable sure that there exists a prime in the range $R_d = \cbrack{ a, a+1, \dots, a+d-1 } \cap S_n$, and then test the primality of these integers.
  
  We assume that the prime numbers are uniformly distributed over the whole interval.
  Then we may assume that for every number $n = R_d$, the probability of this number being a prime is as described in \eqref{eq:probability-of-prime}.
  Then we can do this reasoning for every number in the interval, which again gives us a geometric distribution and we have that the probability of there being a prime in this interval is

  \begin{equation}
    \mathcal{P}_d = 1 - \nbrack{ 1-\mathcal{Q}_n }^{d} = 1 - \nbrack{ 1-\frac{1}{n \ln 2} }^d.
    \label{eq:probability-of-prime-in-rd}
  \end{equation}

  We want to fix $n$ and find a $d$ such that $\mathcal{P}_d$ is large enough for us to be happy.
  Consider an example where we are looking for a $500$ bit prime number.
  That is we consider $n=500$, and say we are happy with a confidence rate of $99\%$.
  A quick numerical computation gives us $d = 1594$, so for this example we would need a width of $d = 1594$ to have a confidence of $99\%$ of finding a prime number.

  However, we will not anymore get a uniformly random prime number.
  Say we do a random search of this interval as described in the previous section, we will more likely find fairly isolated primes.
  As we choose some random $a \in S_n$ first, every prime number will appear in the same amount of intervals, discounting primes in the first or last $d$ numbers, which for large $n$ is negligible.
  But the likelihood of a prime being found is then given by how many other primes there are in this interval.
  If we methodically search for primes in this interval starting with $a$ and then go to $a+1$ and so on, the prime numbers likelihood of being chosen is determined by how long chain of composite numbers are directly in front of this prime number.
  In this case for example the second prime in a twin prime pair is very unlikely of being chosen as there are only two possible randomly chosen numbers that result in this prime being found.


\subsection{Filtering Candidates}\label{sec:filtering-candidates}

  We can now consider a method where we instead of choosing a completely random integer from the interval $S_n$, we rather shrink $S_n$ by removing all integers divisible by small primes.
  By doing this we will remove all even numbers, thus remove half of $S_n$ and have a remaining set with order $1/2 \cdot |S_n|$.
  Further we remove all numbers divisible by $3$, and thus remove a third of all numbers in $S_n$.
  Since divisibility by $2$ and $3$ are independent, we will have a remaining set with order $2/6 \cdot |S_n|$.
  In fact, divisibility by any two prime numbers are independent, so let $\M{P}_B$ denote the set of all prime numbers below a bound $B$.
  Then after our filtration we have a set $\mathcal{F}_B$ of order

  \begin{equation}
    \linebrack{\mathcal{F}_B} = \linebrack{ S_n } \prod_{p\in \M{P}_B} \frac{p-1}{p}.
    \label{eq:order-after-filtration}
  \end{equation}

  Now suppose we chose a random integer $a \in \mathcal{F}_B$, then since we have ruled out a number of non-primes in $S_n$, we will still have the same amount of primes in $\mathcal{F}_B$ as in $S_n$.
  So the probability of $a$ being prime is

  \begin{equation}
    \frac{\pi_n}{\linebrack{\mathcal{F}_B}} = \frac{2^n}{n\ln 2 \cdot \linebrack{S_n}} \prod_{p \in \M{P}_B} \frac{p}{p-1}
    = \frac{1}{n \ln 2} \prod_{p \in \M{P}_B} \frac{p}{p-1}.
    \label{eq:probability-of-prime-in-fb}
  \end{equation}

  Further by our previous logic, we would expect to test

  \begin{equation}
    n \ln 2 \prod_{p \in \M{P}_B} \frac{p-1}{p}
    \label{eq:candidates-teste-with-filtration}
  \end{equation}
  
  candidates.

  So take our usual example of $n = 500$, and let $B = 2000$.
  Then after a, yet again, quick numerical calculation we get an expected number of candidates of $25.5$.
